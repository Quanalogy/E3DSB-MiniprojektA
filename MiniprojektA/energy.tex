\section{Energy} 
\Cref{tab:Energy} shows the energies in signals before and after the Fourier transform. It clearly shows that there is no loss of energy between the original signal and a Fourier transform, but when smoothing and windowing is applied there is a potential loss of energy.
Zero padding does not add or remove any energy from the signal, as it is equivalent to adding \SI{0}{\volt}DC to the signal.
\begin{table}[htb!]
	\centering
	\begin{tabularx}{\textwidth}{p{2cm} | X X X X X}
		& \rotatebox{90}{\textbf{Time Domain $\times\num{e4}$}}   & \rotatebox{90}{\textbf{Frequency Domain $\times\num{e4}$}} & \rotatebox{90}{\textbf{Smooth $\times\num{e3}$}}     & \rotatebox{90}{\textbf{Zero Padding $\times\num{e4}$}}  & \rotatebox{90}{\textbf{Windowing $\times\num{e4}$}} \\
		\hline
		Car Engine  & \num{3,07}	& \num{3,07}	& \num{0,686}  &	\num{3,07}  & \num{1,34}  \\
		
		Windmill	& \num{3,45}	& \num{3,45}	& \num{0,252} & \num{3,45} & \num{1,45} \\
		
		Breaking Wine Glass & \num{0,00666}	& \num{0,00666}	& \num{0,903}	& \num{0,00666}	& \num{0,000724} \\
		
		EKG & \num{0,960}	& \num{0,960}	& \num{0,731}	& \num{0,960}	& \num{0,369} \\
		
		Pop & \num{73,3}	& \num{73,3}	& \num{2,06}	& \num{73,3}	& \num{34,0} \\
		
		Techno & \num{101}	& \num{101}		& \num{1,56}	& \num{101}		& \num{37,2} \\
		
		Heavy Metal & \num{21,1} &	\num{21,1} & \num{1,48}	& \num{21,1}	& \num{9,07} \\
		
		Classical & \num{12,9}	& \num{12,9}	& \num{1,09}	& \num{12,9}	& \num{3,86} \\
	\end{tabularx}
	\caption{Energy of the different signals shown in \si{\joule}.}
	\label{tab:Energy}
\end{table}