% !TEX root = main.tex

\section{Results}

\subsection{FIR vs IIR speed}
The speed of the filters was found by using the built in matlab tic toc function around the filter function as shown in \cref{list:speed}. The filters are attempted to have the same frequency characteristics, which result in different filter orders for the FIR and IIR filters.

\begin{listing}
	\begin{minted}{matlab}
		tic
		yBP = filter(b,a,y);
		toc
	\end{minted}
	\caption{Demonstration of the code used to find the calculation time of the filters.}
	\label{list:speed}
\end{listing}

\begin{table}[!htb]
	\centering
	\caption{Comparing the calculation speed of FIR and IIR filters.}
	\label{tab:speed}
	\begin{tabular}{cc}
		Type of filter & Time(\si{\second}) \\
		\midrule
		FIR low pass & \num{2.275644}\\
		IIR low pass & \num{0.404421}\\
		FIR band pass & \num{2.283198}\\
		IIR band pass & \num{0.463531}
	\end{tabular}
\end{table}

\subsection{FIR and IIR speed dependency of filter order}
To test this a band pass from \SIrange{2048}{8192}{\hertz} will be created using different orders of filters. It is important to note that you cannot compare FIR and IIR filters with the same order in this table, as IIR filters can achieve the same effect on the output as the FIR, but with a much lower filter order. The results are shown in \cref{tab:order_speed}.
\begin{table}[!htb]
	\centering
	\caption{Comparing the calculation speed of FIR and IIR filters with different orders.}
	\label{tab:order_speed}
	\begin{tabular}{ccc}
		Order of filter & FIR time (\si{\second}) & IIR time (\si{\second}) \\
		\midrule
		1		&	\num{0.199696} & \num{0.406552}\\
		5		& \num{0.443048} & \num{0.875347} \\
		10		& \num{0.500542}	 & \num{3.545236}\\
		100		& \num{2.461655}	 & \num{6.413526} \\
		500		& \num{6.044611}	 & \num{21.773430} \\
	\end{tabular}
\end{table}

\subsection{Windows Effect on the FIR Impluse Response}
\Crefrange{fig:FIR_50Hamming}{fig:FIR_Kaiser} demonstrates three different windows and their effect on the impulse response. Comparing the magnitudes response, the Hamming window has a less sharp cutoff at the cutoff frequencies than both the Chebyshev and the Kaiser filter. The Chebyshev filter has a noticable ripple in the passband, whereas the ripple in the Kaiser filter is almost invisible. The phase in all three windows appears almost constant outside of the passband, but as an effect of the Hamming window not having as sharp a cutoff at the cutoff frequencies, it's linear phase is a lot longer than the linear phases for both the Chebyshev and Kaiser windows.
	
\subsection{FIR Filters Effect on the Sound of the Music}
\begin{table}[!htb]
	\begin{tabularx}{\textwidth}{X X}
		Filter & Effect \\
		\midrule
		High pass at \SI{512}{\hertz} & Almost inaudible. \\
		Band pass between \SI{512}{\hertz} and \SI{2048}{\hertz} & Bass and bells stand out a lot. Voice very distorted.\\
		Stop band between \SI{2048}{\hertz} and \SI{8192}{\hertz} & More sound to the bass. Voise less distorted. Small "clicks" during playback. \\
		Band pass between \SI{8192}{\hertz} and \SI{16384}{\hertz} & Bass sounds like a guitar. General sound level is very low. Hi-hat really comes forward.\\
		Low pass at \SI{16384}{\hertz} & Some of the hollow noise from hitting the drums has disappeared.\\
	\end{tabularx}
\end{table}