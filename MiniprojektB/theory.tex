\section{Introduction}
This project investigates the usage of different digital filters as equalizers. The filter types used are Finite Impulse Respones (FIR) and Infinite Impulse Response (IIR) filters.

FIR filters has a finite memory, and only allows an impulse response to affect the system for a FINITE amount of time; such as a lot of digital filters. IIR filters allows the impulse response to affect the system for an INFINITE amount of time, a common example is analog filters containing resistors, capacitors or inductors.

Lav i Matlab en audio equalizer. Equalizeren skal kunne justere niveauet på et indkommende lydsignal individuelt i fem forskellige frekvensbånd fordelt over (og dækkende) det hørbare spektrum med +/- 12 dB. Eksempelvis et bånd fra 0 hertz (eller måske 20 Hz) til 200 Hz. Herefter evt. 200-700 Hz og så fremdeles.

Der skal i opgaven indgå filtre af begge typer (FIR og IIR). Man har frie hænder til valg af designmetode. Vinduesmetoden er fin, hvis det skal foregå manuelt, - ellers er fir1.m og butter.m i Matlab gode bud.

Eksperimenter i opgaven med filterorden og knækfrekvenser. Dokumenter eksperimenterne. Equalizerens samlede impulsrespons og overføringskarakteristik (altså amplitude og fase i frekvens-domænet) skal kunne vises. Dvs. den samlede virkning af de fem filtre fra input til output.



\section{Expectations}