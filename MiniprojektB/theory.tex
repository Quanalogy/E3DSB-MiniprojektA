% !TEX root = main.tex

\section{Introduction}
This project investigates the usage of different digital filters as equalizers. The filter types used are Finite Impulse Respones (FIR) and Infinite Impulse Response (IIR) filters.

FIR filters has a finite memory, and only allows an impulse response to affect the system for a FINITE amount of time.
IIR filters allows the impulse response to affect the system for an INFINITE amount of time.
There are a few key differences between the FIR and IIR.
A FIR filter can run on integer math, whereas IIR requires floating point for the calculations not to blow up. The phase is kept much more stable by using a FIR filter, compared to an IIR filter.
An IIR filter requires fewer coefficients, less memory and is generally faster than a FIR filter.

For this project we are asked to devide the filters into five different bands. Generellay speaking the human hearing is between \SI{20}{\hertz} and \SI{20}{\kilo\hertz} The five bands are taken from \href{https://en.wikipedia.org/wiki/Audio_frequency}{here} and are adjusted to fit our range of five bands in the range of \SIrange{20}{20.000}{\hertz} and are seen in table \ref{tab:FiveBand}.

\begin{table}[b]
	\caption{The five frequency bands used for this miniproject}
	\label{tab:FiveBand}
	\begin{tabularx}{\textwidth}{X X X}
		\textbf{Frequency} \si{\hertz} & \textbf{Octave} & \textbf{Description} \\
		20-512 & 2th to 5th & Lower and upper bass \\
		512-2048 & 6th to 7th & Horn like and tinny quality. Regular speach lies here. \\
		204-8192 & 8th to 9th & labial(-th sounds) and fricative (-s sounds) type of human speach.\\
		8192-16384 & 10th & The sound of bells ringing or sibilance in speach (whistle)\\
		16384-20000 & 11th & The last range of human hearing. \\
	\end{tabularx}	
\end{table}




Lav i Matlab en audio equalizer. Equalizeren skal kunne justere niveauet på et indkommende lydsignal individuelt i fem forskellige frekvensbånd fordelt over (og dækkende) det hørbare spektrum med +/- 12 dB. Eksempelvis et bånd fra 0 hertz (eller måske 20 Hz) til 200 Hz. Herefter evt. 200-700 Hz og så fremdeles.

Der skal i opgaven indgå filtre af begge typer (FIR og IIR). Man har frie hænder til valg af designmetode. Vinduesmetoden er fin, hvis det skal foregå manuelt, - ellers er fir1.m og butter.m i Matlab gode bud.

Eksperimenter i opgaven med filterorden og knækfrekvenser. Dokumenter eksperimenterne. Equalizerens samlede impulsrespons og overføringskarakteristik (altså amplitude og fase i frekvens-domænet) skal kunne vises. Dvs. den samlede virkning af de fem filtre fra input til output.

\section{Expectations}
We are going to do a few experiments comparing FIR and IIR, but also a few investigating the way either of them works.

\begin{table}
	\caption{Comparing equivalent IIR- and FIR filters.}
	\label{tab:IIRvsFIR}
	\begin{tabularx}{\textwidth}{X X X}
		Statement to test	& Expected Result	& Result \\
		\toprule
		The phase is more stable for a FIR filter & The phase of a signal is affected less by running it thorugh a FIR filter than thorugh an IIR filter. & \\
		IIR is faster		& Using tic toc in Matlab during the filtering, IIR filter will yield a lower value. & \\
	\end{tabularx}
\end{table}

\begin{table}
	\caption{Investigation of altering the IIR coefficients.}
	\label{tab:IIRtest}
	\begin{tabularx}{\textwidth}{X X X}
		Statement to test	& Expected Result	& Result \\
		\toprule
		IIR calculation times scales exponentially by increasing coefficients & By multiplying the coefficients by 2, 4 or 8, the calculation time increasses exponentially. & \\
	\end{tabularx}
\end{table}