% !TEX root = main.tex

\section{Introduction}
This project investigates the usage of different digital filters as equalizers. The filter types used are Finite Impulse Respones (FIR) and Infinite Impulse Response (IIR) filters.

FIR filters has a finite memory, and only allows an impulse response to affect the system for a FINITE amount of time.
IIR filters allows the impulse response to affect the system for an INFINITE amount of time.
There are a few key differences between the FIR and IIR.
A FIR filter can run on integer math, whereas IIR requires floating point for the calculations not to blow up.
An IIR filter requires fewer coefficients, less memory and is generally faster than a FIR filter.

For this project we are asked to divide the filters into five different bands. Generelly speaking the human hearing is between \SI{20}{\hertz} and \SI{20}{\kilo\hertz} The five bands are inspired by the Wikipedia article on Audio frequency\footnote{https://en.wikipedia.org/wiki/Audio\_frequency} and are adjusted to fit our range of five bands in the range of \SIrange{0}{22050}{\hertz}. The used values are seen in \cref{tab:FiveBand}.

\begin{table}[h]
	\caption{The frequency bands filtered out. \SI{22050}{\hertz} is the upper limit of frequencies for the analyzed music, which is sampled at \SI{44100}{\hertz}.}
	\label{tab:FiveBand}
	\begin{tabularx}{\textwidth}{X X}
		\textbf{Frequency} (\si{\hertz})	& \textbf{Description} \\
		\midrule
		\numrange{0}{512}		& Rhythm frequencies. \\
		\numrange{512}{2048}	& Horn like and tinny quality. Regular speach lies here. \\
		\numrange{2048}{8192}	& Labial and fricative sounds. \\
		\numrange{8192}{16384}	& Sounds of bells and ringing. \\
		\numrange{16384}{22050} & 	Beyond Brilliance, nebulous sounds approaching and just passing the upper human threshold of hearing. \\
	\end{tabularx}
\end{table}

Lav i Matlab en audio equalizer. Equalizeren skal kunne justere niveauet på et indkommende lydsignal individuelt i fem forskellige frekvensbånd fordelt over (og dækkende) det hørbare spektrum med +/- 12 dB. Eksempelvis et bånd fra 0 hertz (eller måske 20 Hz) til 200 Hz. Herefter evt. 200-700 Hz og så fremdeles.

Der skal i opgaven indgå filtre af begge typer (FIR og IIR). Man har frie hænder til valg af designmetode. Vinduesmetoden er fin, hvis det skal foregå manuelt, - ellers er fir1.m og butter.m i Matlab gode bud.

Eksperimenter i opgaven med filterorden og knækfrekvenser. Dokumenter eksperimenterne. Equalizerens samlede impulsrespons og overføringskarakteristik (altså amplitude og fase i frekvens-domænet) skal kunne vises. Dvs. den samlede virkning af de fem filtre fra input til output.

\section{Expectations}
\label{sec:expectations}

First FIR filters for the bands in \cref{tab:FiveBand} are going to be implemented converted into an .mp4 file for listening. It is expected that the implementation of filters will be most easily heard at frequencies from \SIrange{512}{8192}{\hertz}.
We are going to do a few speed tests. First testing to see how much faster IIR is compared to FIR at comparable output results (not filter order). Then some testing of the impact of filter order on the computation speed of the two filters.
