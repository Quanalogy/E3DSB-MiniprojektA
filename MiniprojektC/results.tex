% !TEX root = main.tex

\section{Results}
In the section "\nameref{sec:analysis}" the data analysis have been done, in this section a discussion upon the effects to the sound samples will be made. First off a highpass filter was applied. The first thing noticed was that the sound levels were lowered a lot. The voice sounded like if someone had recorded it from far away. When put on replay a couple of times, it is also apparent that the two samples have lost some detail in the voice - which would also be expected when filtering out some of the frequencies. None of the two samples sounded like it had "changed gender", it just sounds distorded and far away. The same applies for the bandpass and the male voice, the female does sound a bit different with the bandpass, it actually sounds like her voice is even further away than with the highpass filter. A better apporoach to doing this would be to have done some shifting after doing the fourier transform, because it is the frequencies that decide if we would think it as a male or female - the males tend to have lower frequencies, hence the female signal should not filter out the signals outside the male but should have moved the frequencies down to were the male frequencies lie. 

From testing the filters (IIR and FIR) it became evident that the order of the filters were important. The FIR bandpass filter was first implemented with an order 5 and 10, which did not affect the signal as expected. When the filter order of the FIR filter was changed to 100, the outputs were as expected. The IIR bandpass filters were implemented as a fifth order filter, when applied this gave the expected results, hence an experience from this is that the FIR filter need a greater order to function as expected compared to the IIR, while the IIR filter can become unstable when using too high orders. When using higher orders of IIR filters, we found the poles to appear outside of the unit circle, making the filter unstable