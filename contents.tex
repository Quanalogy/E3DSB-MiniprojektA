\section{Discussion}

In this mini project the analysis has demonstrated Parceval's theorem that the Fourier transform of a time signal will not change the energy of the signal, which according to our data (see \cref{tab:Energy}) also applies to zero padding. The Smooth function does change the energy in the signal, which makes sense because it makes avarage values of a set of 5 bins and then continues, hence a more avarage signal, where large differences would be more "smooth". The window function do also change the energy of the signal. This does also makes sense the Hamming function in use is seen in \cref{eq:hamming} which applied to the general function in \cref{eq:window} at best will give an factor of $w(n)=0.54$.

\section{Conclusion}

This mini project has given insight into the different kinds of ways you can manipulate a signal, and it's fourier transform, to gain access to data which was otherwise obscured in the pure fft. Noisy signals gives a clearer output if the fft is smoothed, as seen especially in the car, windmill and musical analysis. Zero padding allows for a more precise determination of a specific frequency, such as the frequency of the EKG. Lastly windowing should help getting rid of leakage, making peaks more prominent.

Energywise it has been shown that the original signal, the fourier transform and a fourier transform of a zero padded signal contains the same amount of energy, whereas smoothing and windowing will remove some of the energy.