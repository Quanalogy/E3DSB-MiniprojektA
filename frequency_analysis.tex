\section{Frequency analysis}
\label{sec:analysis}

In this following section an analysis will be made of the sound made by a car motor (figure \ref{fig:car}), windmill (figure \ref{fig:windmill}), EKG (figure \ref{fig:ekg}), breaking wine glass (figure \ref{fig:glassBreaking}) and four different genres of music. 
The approach of this analysis will be to plot the signal in the time domain, the fourier transformed signal and then some applied functions described in section \ref{sec:theory}.


\subsection{Car engine}

The original signal from the car engine is seen in figure \ref{fig:car_time}, while the Fourier transformed signal is seen in figure \ref{fig:car_freq}. The following are also Fourier transformed, but have applied smoothing (figure \ref{fig:car_smooth}), zeropadding (figure \ref{fig:car_zero}) and hamming (figure \ref{fig:car_window}). Notice the magnitude difference from the Fourier transformed to the smoothing.

\begin{figure}[htb!]
	\centering
	\subcaptionbox{Time Domain.\label{fig:car_time}}
	{\includegraphics[width=0.45\linewidth]{code/Car_figure1.png}}
	\subcaptionbox{Frequency Domain.\label{fig:car_freq}}
	{\includegraphics[width=0.45\linewidth]{code/Car_figure2.png}}
	\subcaptionbox{Smoothed fft.\label{fig:car_smooth}}
	{\includegraphics[width=0.45\linewidth]{code/Car_figure3.png}}
	\subcaptionbox{fft of zero padded original.\label{fig:car_zero}}
	{\includegraphics[width=0.45\linewidth]{code/Car_figure4.png}}
	\subcaptionbox{fft of windowed original.\label{fig:car_window}}
	{\includegraphics[width=0.45\linewidth]{code/Car_figure5.png}}
	\caption{Analysis of the sound of a car.}\label{fig:car}
\end{figure}


\subsection{Noise from a Windmill}
The original signal from the windmill is seen plotted in the timedomain in figure \ref{fig:windmill_time}, while the fourier transformed signal is seen in figure \ref{fig:windmill_freq}.
\Cref{fig:windmill_smooth,fig:windmill_zero,fig:windmill_window}, shows the smoothed fft, how zero padding affects the fft, and how windowing affects the fft.

\begin{figure}[htb!]
	\centering
	\subcaptionbox{Time Domain.\label{fig:windmill_time}}
	{\includegraphics[width=0.45\linewidth]{code/Windmill_figure1.png}}
	\subcaptionbox{Frequency Domain.\label{fig:windmill_freq}}
	{\includegraphics[width=0.45\linewidth]{code/Windmill_figure2.png}}
	\subcaptionbox{Smoothed fft.\label{fig:windmill_smooth}}
	{\includegraphics[width=0.45\linewidth]{code/Windmill_figure3.png}}
	\subcaptionbox{fft of zero padded original.\label{fig:windmill_zero}}
	{\includegraphics[width=0.45\linewidth]{code/Windmill_figure4.png}}
	\subcaptionbox{fft of windowed original.\label{fig:windmill_window}}
	{\includegraphics[width=0.45\linewidth]{code/Windmill_figure5.png}}
	\caption{Analysis of the sound of a windmill.}\label{fig:windmill}
\end{figure}


\subsection{EKG}
The original signal from the EKG is seen plotted in the timedomain in figure \ref{fig:ekg_time}, while the fourier transformed signal is seen in figure \ref{fig:ekg_freq}.
\Cref{fig:ekg_smooth,fig:ekg_zero,fig:ekg_window}, shows the smoothed fft, how zero padding affects the fft, and how windowing affects the fft.


\begin{figure}[htb!]
	\centering
	\subcaptionbox{Time Domain.\label{fig:ekg_time}}
	{\includegraphics[width=0.45\linewidth]{code/ekg_figure1.png}}
	\subcaptionbox{Frequency Domain.\label{fig:ekg_freq}}
	{\includegraphics[width=0.45\linewidth]{code/ekg_figure2.png}}
	\subcaptionbox{Smoothed fft.\label{fig:ekg_smooth}}
	{\includegraphics[width=0.45\linewidth]{code/ekg_figure3.png}}
	\subcaptionbox{fft of zero padded original.\label{fig:ekg_zero}}
	{\includegraphics[width=0.45\linewidth]{code/ekg_figure4.png}}
	\subcaptionbox{fft of windowed original.\label{fig:ekg_window}}
	{\includegraphics[width=0.45\linewidth]{code/ekg_figure5.png}}
	\caption{Analysis of the sound of an EKG.}\label{fig:ekg}
\end{figure}

\subsection{Breaking Wine Glass}

The original signal from the breaking glass is seen plotted in the timedomain in figure \ref{fig:glassBreaking_time}, while the fourier transformed signal is seen in figure \ref{fig:glassBreaking_freq}.
\Cref{fig:glassBreaking_smooth,fig:glassBreaking_zero,fig:glassBreaking_window}, shows the smoothed fft, how zero padding affects the fft, and how windowing affects the fft.

\begin{figure}[htb!]
	\centering
	\subcaptionbox{Time Domain.\label{fig:glassBreaking_time}}
	{\includegraphics[width=0.45\linewidth]{code/glassBreaking_figure1.png}}
	\subcaptionbox{Frequency Domain.\label{fig:glassBreaking_freq}}
	{\includegraphics[width=0.45\linewidth]{code/glassBreaking_figure2.png}}
	\subcaptionbox{Smoothed fft.\label{fig:glassBreaking_smooth}}
	{\includegraphics[width=0.45\linewidth]{code/glassBreaking_figure3.png}}
	\subcaptionbox{fft of zero padded original.\label{fig:glassBreaking_zero}}
	{\includegraphics[width=0.45\linewidth]{code/glassBreaking_figure4.png}}
	\subcaptionbox{fft of windowed original.\label{fig:glassBreaking_window}}
	{\includegraphics[width=0.45\linewidth]{code/glassBreaking_figure5.png}}
	\caption{Analysis of the sound of a glas breaking.}\label{fig:glassBreaking}
\end{figure}


\subsection{Music}

\paragraph{Pop}
Michael Jackson - Thriller

The original signal from the song "Thriller" is seen plotted in the timedomain in figure \ref{fig:pop_time}, while the fourier transformed signal is seen in figure \ref{fig:pop_freq}.
\Cref{fig:pop_smooth,fig:pop_zero,fig:pop_window}, shows the smoothed fft, how zero padding affects the fft, and how windowing affects the fft.


\begin{figure}[htb!]
	\centering
	\subcaptionbox{Time Domain.\label{fig:pop_time}}
	{\includegraphics[width=0.45\linewidth]{code/Pop_figure1.png}}
	\subcaptionbox{Frequency Domain.\label{fig:pop_freq}}
	{\includegraphics[width=0.45\linewidth]{code/Pop_figure2.png}}
	\subcaptionbox{Smoothed fft.\label{fig:pop_smooth}}
	{\includegraphics[width=0.45\linewidth]{code/Pop_figure3.png}}
	\subcaptionbox{fft of zero padded original.\label{fig:pop_zero}}
	{\includegraphics[width=0.45\linewidth]{code/Pop_figure4.png}}
	\subcaptionbox{fft of windowed original.\label{fig:pop_window}}
	{\includegraphics[width=0.45\linewidth]{code/Pop_figure5.png}}
	\caption{Analysis of the sound of Pop.}\label{fig:pop}
\end{figure}

\paragraph{Techno}
Matador - The Enemy ft Felix Da Housecat (Original Mix)

The original signal from the song "The Enemy" is seen plotted in the timedomain in figure \ref{fig:techno_time}, while the fourier transformed signal is seen in figure \ref{fig:techno_freq}. \Cref{fig:techno_smooth,fig:techno_zero,fig:techno_window}, shows the smoothed fft, how zero padding affects the fft, and how windowing affects the fft..

\begin{figure}[htb!]
	\centering
	\subcaptionbox{Time Domain.\label{fig:techno_time}}
	{\includegraphics[width=0.45\linewidth]{code/Techno_figure1.png}}
	\subcaptionbox{Frequency Domain.\label{fig:techno_freq}}
	{\includegraphics[width=0.45\linewidth]{code/Techno_figure2.png}}
	\subcaptionbox{Smoothed fft.\label{fig:techno_smooth}}
	{\includegraphics[width=0.45\linewidth]{code/Techno_figure3.png}}
	\subcaptionbox{fft of zero padded original.\label{fig:techno_zero}}
	{\includegraphics[width=0.45\linewidth]{code/Techno_figure4.png}}
	\subcaptionbox{fft of windowed original.\label{fig:techno_window}}
	{\includegraphics[width=0.45\linewidth]{code/Techno_figure5.png}}
	\caption{Analysis of the sound of Techno.}\label{fig:techno}
\end{figure}

\paragraph{Heavy Metal}
Black Sabbath - Iron Man

The original signal from the song "Iron Man" is seen plotted in the timedomain in \cref{fig:heavy_time}, while the fourier transformed signal is seen in \cref{fig:heavy_freq}. \Cref{fig:heavy_smooth,fig:heavy_zero,fig:heavy_window}, shows the smoothed fft, how zero padding affects the fft, and how windowing affects the fft.

\begin{figure}[htb!]
	\centering
	\subcaptionbox{Time Domain.\label{fig:heavy_time}}
	{\includegraphics[width=0.45\linewidth]{code/HeavyMetal_figure1.png}}
	\subcaptionbox{Frequency Domain.\label{fig:heavy_freq}}
	{\includegraphics[width=0.45\linewidth]{code/HeavyMetal_figure2.png}}
	\subcaptionbox{Smoothed fft.\label{fig:heavy_smooth}}
	{\includegraphics[width=0.45\linewidth]{code/HeavyMetal_figure3.png}}
	\subcaptionbox{fft of zero padded original.\label{fig:heavy_zero}}
	{\includegraphics[width=0.45\linewidth]{code/HeavyMetal_figure4.png}}
	\subcaptionbox{fft of windowed original.\label{fig:heavy_window}}
	{\includegraphics[width=0.45\linewidth]{code/HeavyMetal_figure5.png}}
	\caption{Analysis of the sound of Heavy Metal.}\label{fig:heavy}
\end{figure}

\paragraph{Classical}
Vivaldi - La Follia

The original signal from "La Follia" is seen plotted in the timedomain in \cref{fig:klassisk_time}, while the fourier transformed signal is seen in \cref{fig:klassisk_freq}. \Cref{fig:klassisk_smooth,fig:klassisk_zero,fig:klassisk_window}, shows the smoothed fft, how zero padding affects the fft, and how windowing affects the fft.

\begin{figure}[htb!]
	\centering
	\subcaptionbox{Time Domain.\label{fig:klassisk_time}}
	{\includegraphics[width=0.45\linewidth]{code/Classical_figure1.png}}
	\subcaptionbox{Frequency Domain.\label{fig:klassisk_freq}}
	{\includegraphics[width=0.45\linewidth]{code/Classical_figure2.png}}
	\subcaptionbox{Smoothed fft.\label{fig:klassisk_smooth}}
	{\includegraphics[width=0.45\linewidth]{code/Classical_figure3.png}}
	\subcaptionbox{fft of zero padded original.\label{fig:klassisk_zero}}
	{\includegraphics[width=0.45\linewidth]{code/Classical_figure4.png}}
	\subcaptionbox{fft of windowed original.\label{fig:klassisk_window}}
	{\includegraphics[width=0.45\linewidth]{code/Classical_figure5.png}}
	\caption{Analysis of the sound of Classical.}\label{fig:klassisk}
\end{figure}

\subsubsection{Conclusion}

For the car sound we see a peak in what appears to be \SI{300}{\hertz}, while the windmill is \SI{90}{\hertz}, the EKG in about \SI{5000}{\hertz}, the glass in about \SI{1000}{\hertz}. 

For the songs the popsong has a peak in about \SI{90}{\hertz}, the Techno \SI{90}{\hertz}, the heavy metal \SI{90}{\hertz} and the classical in about \SI{1000}{\hertz}. The first three songs are very much alike when looking at the peak value, but the expression of the songs are quite different. For the pop song a very steep shift from \SI{60}{\hertz} with a magnitude of about \SI{50}{\decibel}, to the peak of \SI{90}{\hertz} of about \SI{100}{\decibel}. The Techno song shifts from \SI{50}{\hertz} with a magnitude of \SI{50}{\decibel} to the peak of \SI{90}{\hertz} with a magnitude of \SI{110}{\decibel}. The Heavy Metal song has a more steady look, with a shift at \SI{80}{\hertz} with a magnitude of \SI{60}{\decibel} to \SI{90}{\hertz} with a magnitude of \SI{80}{\decibel}. From this data it is apparent that eventhough the three songs has a similar peak value, the signal itself has a way higher magnitude in the lower frequency in the Heavy Metal and Techno song, than the pop song have.
