\section{Using build-in Matlab functions}
\label{sec:functions}
Before digging into the implementation of the analysis, a quick introduction to the functions from matlab that are in use.
\begin{longtabu}{>{\em}X l X}
\normalfont{\textbf{Function}} & Output & \textbf{Description} \\
\hline
audioread(filename) & [y, Fs] & Read an audiofile and put the contents into two outputs. "y" is the sample data, while "Fs" is the sample rate for the data. \\
fft(y) & Y & Computes the discrete Fourier transform (DFT) of y, using a fast Fourier transform algorithm. \\
length(x) & L & Return the length of the largest array dimension in x.  \\
figure(n) & none & Set a new figure for eg a plot to be saved inside. "n" is the figurenumber \\
plot(x, y, c) & none & Plot a x-sequence "x", y-sequence "y" and a color/representation parameter "c". \\
title(titleName) & none & Sets the title of a figure. \\
xlabel(labelName) & none & Sets the label for the x-axis. \\
ylabel(labelName) & none & Sets the label for the y-axis. \\
hold on/off & none & Holds a plot inside a figure when "on" is set, stops the holding of the plot when hold off is set. A new plot after a hold off creates a new figure. \\
grid on & none & set a grid for the plot. \\
max(x) & M & Return the largest elements in an array. \\
mean(x) & m & Return the avarage or mean value of an array. \\
smooth(x, [span], [method]) & xx & Smoothing the data with a default span value of "5" and a default method of "moving". Setting the span and method is optional.\\
semilogx(Y) & h & Plot data on logarithmic scale on the x-axis. \\
legend(label1, ..., labelN, 'Location', 'Value') & none & Creates a box for description of one or more plots, color seperated. The value of Location can be described with nord, west, south, east - or in combination. In the following section the parameter "southwest" is used. \\
cat(dimension, Array1, ..., ArrayN) & C & Concatenate arrays along a specific dimension.  \\
hamming(L) & w & return a L-point symmetric Hamming window in the column vector w. The formula $w(n)$ is described in \cref{sec:windowing}. \\
sum(A) & S & Return the sum of an array, can optionally take another parameter which handle the dimension of the array. \\
saveas(fig, filename, fileformat) & none & Saves a file with a figure "fig", the filename "filename" and the fileformat "fileformat", at the location of the file being executed. \\
xlim([xmin, xmax]) & none & Sets the start and end value on the x-axis of a figure. \\
\end{longtabu}
