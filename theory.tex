\section{Theory}

In this miniproject, the application of Fourier, dB, smoothing, zeropadding, windowing and energy will be performed, but first a quick glanse at the theory behind it all.

\subsection{Fourier}

Fourier is a method for getting a signal from the time domain into the frequency domain. 
For a fourier function a signal go from $x(t) \rightarrow X(f)$. Fourier is defined by:

\begin{equation}
X(f) = \int_{-\infty}^{\infty} x(t)e^{-j2\pi ft}dt
\end{equation}

where $x(t)$ is some continous signal \cite[p. 53]{DSP}. 
When a fourier transform is needed in MATLAB, the "fft" function can be used. 
"fft" stands for "fast fourier transform". 
The "fft" function computes the discrete fourier transform (DFT) of some signal by using a fast fourier transform algorithm \cite[fft]{MATLAB_DOC}. 
By default "fft" only take one parameter, which is the signal:

\begin{minted}{matlab}
	Y = fft(y)
\end{minted}

\subsection{The dB-scale}

The \SI{}{\decibel} scale is used for plotting signals in the frequency domain and is deducted from the bel scale. 
The bel scale is defined by the base 10 logarithm of the difference between two powers\cite[Appendix E]{DSP}

\[
\log_{10}\frac{P1}{P2}\SI{}{\bel}
\] 

The bel scale itself can be very big which led to the unit decibel, that is one tenth of a bel:

\begin{equation}
10\log_{10}\frac{P1}{P2}\SI{}{\decibel}
\end{equation}

This can also be used in combination of a function, which when applied to $X(m)$ gives:

\begin{equation}
X_{\SI{}{\decibel}}(m) = 10\log_{10}(|X(m)|^2)\SI{}{\decibel} = 20\log_{10}(|X(m)|)\SI{}{\decibel}
\end{equation}

\subsection{Smoothing}

When using Matlab, the "smooth" function take a column vector and smoothen the data by making avarages of a defined span, which by default is 5 datapoints\cite[smooth]{MATLAB_DOC}. 
For this project the default value has been used, hence the prototype for the function is:

\begin{minted}{matlab}
	Y_smooth = smooth(Y)
\end{minted}
If another span value is needed the prototype for smooth is:


\begin{minted}{matlab}
	Y_smooth = smooth(Y, span_value)
\end{minted}

The smooth function is great for getting an better overview of data with many different values, becuase of the avarages it makes. 
Take the car sound as an example on figure \ref{fig:Car_figure5:2}, here the red is the fourier transformed function, while the blue is the smoothened function of the red one. 
It is apparent that the blue signal haven't got the crazy magnitude of the red, but resemples the same data - we can still see the general nature of the signal, it is just represented in a more human readable way.

\subsection{Zero padding}

Zero padding is a way of getting more frequency bins hence getting a better visual representation of the signal. 
To do Zeropadding, a series of 0'es is appended to the end of the signal, afterwards the discrete fourier transform is performed. 
Example code of zeropadding the signal from a car motor in Matlab is seen below:

\begin{minted}{matlab}
	[y,Fs] = audioread('./car.wav');
	y_zpad = cat(1, y, zeros(N, 1));
	Y_zpad = fft(y_zpad);
\end{minted}

\subsection{Windowing}
\label{sec:windowing}
Window functions are made for reducing leakage, which can be induced during DFT. If we have an input sequence of $x(n)$ and a window function $w(n)$, where $N$ is the sum of all samples, $n$ is a given sample, the formula of any window function is

\begin{equation}
X_w(m) = \sum_{n=0}^{N-1} w(n)\cdot x(n)e^{\frac{-j2\pi nm}{N}}
\end{equation}

In section \ref{sec:analysisOfSignals} the Hamming window is used, which have the following formula

\begin{equation}
w(n) = 0.54-0.46\cos\left(\frac{2\pi n}{N}\right)
\end{equation}

\subsection{Energy}

Later on when the different functions just described in this section is applyied, the energy of a signal is not always the same before and after. When doing the Fourier transform, the Parceval's theorm ensures that the energy before and after are the same, while when applying smooth, this is not the case. 